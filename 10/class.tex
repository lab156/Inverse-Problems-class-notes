%25 de febrero
We continue looking at heuristics of choosing Regularization parameters.

Considering $L$ to define a seminorm. Still looking at the problem:
$$ \min_x \{|Ax-b|^2 + l^2 |Lx|^2 \}$$
Let $A$ and $L$ be such that $\nuc(A)\cap \nuc(L)=\{0\}$. If $z\in \nuc(A)$ then $Az=0$ and then $Lz\neq 0$.

\begin{ddef}[Generalized Singular Value Decomposition]
The matrix $A$ is factorized into:
\begin{gather}
A = U\begin{pmatrix} \Sigma & 0 \\ 0 & I \end{pmatrix} X ^{-1}    \\
L= V[M | 0 ] X ^{-1}    
\end{gather}
$A_{m\times n}$ and $L_{p\times n}$ 
\begin{itemize}
\item  $U_{m\times m}$ is ortogonal.
\item $V_{p\times p}$ is orthogonal.
\item  $X ^{-1}_{n\times n} $ and has ortogonal columns i.e. $X^TA^T A X=I_{n\times n}$.
\item $0\leq \sigma_1 \leq \sigma_2 \leq \ldots \leq \sigma_p \leq 1$
\item $1\geq \mu_1 \geq \ldots \geq \mu_p > 0 $
\item $\sigma_i^2 + \mu_i^2=1$ and the \textbf{generalized singular values} of $AL$ are $\gamma_i = \sigma_i/\mu_i$.
\end{itemize}
\end{ddef}

Furthermore:
\begin{gather*}
XA^TAX= \begin{bmatrix}
\Sigma^2 & 0 \\
0     & I 
\end{bmatrix} \\
LX = V[M\, 0] \\
X^TL^T = \begin{bmatrix}
M^T \\ 0 \end{bmatrix} V^T 
X^T L^T L X = \begin{bmatrix}
M^T \\ 0 \end{bmatrix} V^T V \begin{bmatrix} M & 0 \end{bmatrix} 
\end{gather*}

\begin{examples}
Consider a system of equations with $x$ length of 2000 and 20 equations. In this case $L$ is going to play a big role. For example, $L$ can be used to control the magnitude of the first derivative. A matrix $L_1$ that is a backward estimate for the first derivative of $x$ has the form is:
\begin{equation}
L_1 = \begin{bmatrix} 1 & -1  &   &       && \\
                          &  1  & -1&       && \\
                          &     &   & \ddots&& \\
                          &     &   &       &1&-1\\
                          &     &   &       & & 1
\end{bmatrix} 
\end{equation}
This matrix is of size $(n-1)\times n$ and $\nuc(L_1) = \{[c,c,\ldots,c]\}= \{c\mathbf{1} \}$

An estimate for the second derivative is:
\begin{equation}
L_1 = \begin{bmatrix} -1 & 2  & -1  &       && \\
                          &  -1  & 2&-1       && \\
                          &     &   & \ddots&& \\
                          &     &   &       -1&2&1
\end{bmatrix} 
\end{equation}
The size in this case is $(n-2)\times n$ and of course the kernel is $\ker(L_2)=\{c\mathbf{1}, c[i] \}$ that correspond to the constant and linear functions.
\end{examples}
                         
